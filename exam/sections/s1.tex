\section*{Problem 1}
\subsection*{Periodicity}
\begin{description}
    \item[i)] $x(t) = 3\cos(3t + \pi/3)$ is periodic as it is a simple sinusoid with $T=\frac{2\pi}{3}$.
    \item[ii)] $x(t) = 3\cos^2(3t + \pi/3)$ is also periodic. To see this note that $y(t) = 3\cos(3t + \pi/3)$ is periodic as before. Then that $y(t + \frac{\pi}{3}) = -y(t)$. but since we take the square $x(t) = y^2(t) = y^2(t + \frac{\pi}{3})$ it now is periodic with half the period of $y(t)$, namely $T=\frac{\pi}{3}$.
\item[iii)] $x(t) = \cos(\frac{\pi}{2}t) + \cos(\frac{1}{2}t)$ is a sum of sinusoids with periods of $T_1 = 4$ and $T_2 = 4\pi$. $x(t)$ is periodic if there exist two integers $n,m$ such that $n\cdot T_1 = m\cdot T_2$. This is not possible as the fraction of $n/m \propto \pi$, which is irrational. Hence, no such integer numbers exist.

\item[iv)] $x(t) = e^{j\pi t}$ is trivially periodic, as it is a simple complex exponential and thus by Eulers rule, a sum of sinusoids of equal periods.
\end{description}

\subsection*{System properties}
\begin{equation}
    y(t) = x(t)x(t+2).
\end{equation}
\begin{description}
    \item[memoryless?] No, as the system depends on future values.
    \item[causal?] No, as the system depends on future values.
    \item[stable?] Yes. Let $x(t)$ be bounded by $|x(t)| < M < \infty$. Then 
    $|y(t)| = |x(t) \cdot x(t+2)| \leq M^2 < \infty$. Hence, $y(t)$ is bounded
    by $M^2$ for $M$ bounded input and thus stable.
    \item[TI] Yes. To be time invarient we need 
    \begin{equation}
        y(t + t_0) = x(t+t_0)x(t+2+t_0),
    \end{equation}
    which can be seen by insertion.

    \item[linear] No. Let $x(t) = a\cdot x_0(t) + b\cdot x_1(t)$. Then 

    \begin{align}
        x(t)x(t+2) &= (a\cdot x_0(t) + b\cdot x_1(t))
        (a\cdot x_0(t+2) + b\cdot x_1(t+2))\\
            &= a^2\cdot x_0(t)x_0(t+2) + b^2\cdot x_1(t)x_1(t+2) \nonumber\\
        &+ ab\cdot (x_0(t)x_1(t+2) + x_0(t+2)x_1(t))\\ 
            &\neq a\underbrace{\cdot x_0(t)x_0(t+2)}_{y(t)\mid _{x(t)=x_0(t)}}
        + b\cdot \underbrace{x_1(t)x_1(t+2)}_{y(t)\mid_{x(t) = x_1(t)}}.
    \end{align}
\end{description}

\begin{equation}
    y[n] = max\{x[n], x[n+1]\}.
\end{equation}

\begin{description}
    \item[memoryless?] No, as the system depends on future values.
    \item[causal?] No, as the system depends on future values.
    \item[stable?] Yes. Let $x(t)$ be bounded by $M$ as before. Then by definition 
    $y(t)$ is also bounded by $M$ and thus it is stable.
    \item[TI?] Yes. Any values we input will shift the output by the same. As 
    we compare to the one-setp-ahead value, we will always cover the same pairings. Different integer jumps could produce different pairing depending on time 
    delays of the imput and could be time variant.
    \item[Linear] No. Generally\footnote{There are exceptions, but that's not
    the point.}
    \begin{align}
        &a\cdot max\{x_0[n], x_0[n+1]\} + b\cdot max\{x_1[n],
        x_1[n+1]\}\nonumber\\
        &\neq max\{ax_0[n] + bx_1[n], ax_0[n+1] + bx_1[n+1]\}
    \end{align}
Consider e.g.
\begin{align}
    x_0[n] &= [\dots, 1, 2, 1, 2, \cdots]\\
    x_1[n] &= [\dots, 1, -1, 1, -1, \cdots]\\ 
\Rightarrow x_0[n]+x_1[n] &= [\dots, 2, 1, 2, 1, \dots]\text{ and }\\
    a&=b=1.
\end{align}

Then 
\begin{align}
    y_0[n] &= [\dots, 2, 2, 2, \dots]\\
    y_1[n] &= [\dots, 1, 1, 1, \dots]\\
\Rightarrow y_0[n]+y_1[n] &= [\dots, 3, 3, 3, \dots],
\end{align}

but
\begin{align}
    y_{x_0 + x_1}[n] &= [\dots, 2, 2, 2, \dots]\\
    &\neq y_0[n]+y_1[n].
\end{align}
\end{description}

